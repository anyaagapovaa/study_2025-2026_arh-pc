% Options for packages loaded elsewhere
% Options for packages loaded elsewhere
\PassOptionsToPackage{unicode}{hyperref}
\PassOptionsToPackage{hyphens}{url}
%
\documentclass[
  english,
  russian,
  12pt,
  a4paper,
  DIV=11,
  numbers=noendperiod]{scrreprt}
\usepackage{xcolor}
\usepackage{amsmath,amssymb}
\setcounter{secnumdepth}{5}
\usepackage{iftex}
\ifPDFTeX
  \usepackage[T1]{fontenc}
  \usepackage[utf8]{inputenc}
  \usepackage{textcomp} % provide euro and other symbols
\else % if luatex or xetex
  \usepackage{unicode-math} % this also loads fontspec
  \defaultfontfeatures{Scale=MatchLowercase}
  \defaultfontfeatures[\rmfamily]{Ligatures=TeX,Scale=1}
\fi
\usepackage{lmodern}
\ifPDFTeX\else
  % xetex/luatex font selection
\fi
% Use upquote if available, for straight quotes in verbatim environments
\IfFileExists{upquote.sty}{\usepackage{upquote}}{}
\IfFileExists{microtype.sty}{% use microtype if available
  \usepackage[]{microtype}
  \UseMicrotypeSet[protrusion]{basicmath} % disable protrusion for tt fonts
}{}
\usepackage{setspace}
% Make \paragraph and \subparagraph free-standing
\makeatletter
\ifx\paragraph\undefined\else
  \let\oldparagraph\paragraph
  \renewcommand{\paragraph}{
    \@ifstar
      \xxxParagraphStar
      \xxxParagraphNoStar
  }
  \newcommand{\xxxParagraphStar}[1]{\oldparagraph*{#1}\mbox{}}
  \newcommand{\xxxParagraphNoStar}[1]{\oldparagraph{#1}\mbox{}}
\fi
\ifx\subparagraph\undefined\else
  \let\oldsubparagraph\subparagraph
  \renewcommand{\subparagraph}{
    \@ifstar
      \xxxSubParagraphStar
      \xxxSubParagraphNoStar
  }
  \newcommand{\xxxSubParagraphStar}[1]{\oldsubparagraph*{#1}\mbox{}}
  \newcommand{\xxxSubParagraphNoStar}[1]{\oldsubparagraph{#1}\mbox{}}
\fi
\makeatother


\usepackage{longtable,booktabs,array}
\usepackage{calc} % for calculating minipage widths
% Correct order of tables after \paragraph or \subparagraph
\usepackage{etoolbox}
\makeatletter
\patchcmd\longtable{\par}{\if@noskipsec\mbox{}\fi\par}{}{}
\makeatother
% Allow footnotes in longtable head/foot
\IfFileExists{footnotehyper.sty}{\usepackage{footnotehyper}}{\usepackage{footnote}}
\makesavenoteenv{longtable}
\usepackage{graphicx}
\makeatletter
\newsavebox\pandoc@box
\newcommand*\pandocbounded[1]{% scales image to fit in text height/width
  \sbox\pandoc@box{#1}%
  \Gscale@div\@tempa{\textheight}{\dimexpr\ht\pandoc@box+\dp\pandoc@box\relax}%
  \Gscale@div\@tempb{\linewidth}{\wd\pandoc@box}%
  \ifdim\@tempb\p@<\@tempa\p@\let\@tempa\@tempb\fi% select the smaller of both
  \ifdim\@tempa\p@<\p@\scalebox{\@tempa}{\usebox\pandoc@box}%
  \else\usebox{\pandoc@box}%
  \fi%
}
% Set default figure placement to htbp
\def\fps@figure{htbp}
\makeatother



\ifLuaTeX
\usepackage[bidi=basic,provide=*]{babel}
\else
\usepackage[bidi=default,provide=*]{babel}
\fi
% get rid of language-specific shorthands (see #6817):
\let\LanguageShortHands\languageshorthands
\def\languageshorthands#1{}


\setlength{\emergencystretch}{3em} % prevent overfull lines

\providecommand{\tightlist}{%
  \setlength{\itemsep}{0pt}\setlength{\parskip}{0pt}}



 
\usepackage[backend=biber,langhook=extras,autolang=other*]{biblatex}
\addbibresource{bib/cite.bib}

\usepackage[]{csquotes}

\usepackage{indentfirst}
\usepackage{float}
\floatplacement{figure}{H}
\IfFileExists{plex-otf.sty}{
  %% Full TeXlive
  \usepackage[math,RM={Scale=0.94},SS={Scale=0.94},SScon={Scale=0.94},TT={Scale=MatchLowercase,FakeStretch=0.9},DefaultFeatures={Ligatures=Common}]{plex-otf}
}{
  %% TinyTeX
  \usepackage{libertine}
}
\KOMAoption{captions}{tableheading}
\makeatletter
\@ifpackageloaded{caption}{}{\usepackage{caption}}
\AtBeginDocument{%
\ifdefined\contentsname
  \renewcommand*\contentsname{Содержание}
\else
  \newcommand\contentsname{Содержание}
\fi
\ifdefined\listfigurename
  \renewcommand*\listfigurename{Список иллюстраций}
\else
  \newcommand\listfigurename{Список иллюстраций}
\fi
\ifdefined\listtablename
  \renewcommand*\listtablename{Список таблиц}
\else
  \newcommand\listtablename{Список таблиц}
\fi
\ifdefined\figurename
  \renewcommand*\figurename{Рисунок}
\else
  \newcommand\figurename{Рисунок}
\fi
\ifdefined\tablename
  \renewcommand*\tablename{Таблица}
\else
  \newcommand\tablename{Таблица}
\fi
}
\@ifpackageloaded{float}{}{\usepackage{float}}
\floatstyle{ruled}
\@ifundefined{c@chapter}{\newfloat{codelisting}{h}{lop}}{\newfloat{codelisting}{h}{lop}[chapter]}
\floatname{codelisting}{Список}
\newcommand*\listoflistings{\listof{codelisting}{Листинги}}
\makeatother
\makeatletter
\makeatother
\makeatletter
\@ifpackageloaded{caption}{}{\usepackage{caption}}
\@ifpackageloaded{subcaption}{}{\usepackage{subcaption}}
\makeatother
\usepackage{bookmark}
\IfFileExists{xurl.sty}{\usepackage{xurl}}{} % add URL line breaks if available
\urlstyle{same}
\hypersetup{
  pdftitle={Отчёт по лабораторной работе №3},
  pdfauthor={Агапова Анна Антоновна},
  pdflang={ru-RU},
  hidelinks,
  pdfcreator={LaTeX via pandoc}}


\title{Отчёт по лабораторной работе №3}
\usepackage{etoolbox}
\makeatletter
\providecommand{\subtitle}[1]{% add subtitle to \maketitle
  \apptocmd{\@title}{\par {\large #1 \par}}{}{}
}
\makeatother
\subtitle{Архитектура компьютера}
\author{Агапова Анна Антоновна}
\date{}
\begin{document}
\maketitle

\renewcommand*\contentsname{Содержание}
{
\setcounter{tocdepth}{1}
\tableofcontents
}
\listoffigures
\listoftables

\setstretch{1.5}
\chapter{Цель
работы}\label{ux446ux435ux43bux44c-ux440ux430ux431ux43eux442ux44b}

Освоение процедуры компиляции и сборки программ, написанных на
ассемблере NASM.

\chapter{Выполнение лабораторной
работы}\label{ux432ux44bux43fux43eux43bux43dux435ux43dux438ux435-ux43bux430ux431ux43eux440ux430ux442ux43eux440ux43dux43eux439-ux440ux430ux431ux43eux442ux44b}

1.Создаю каталог для работы с программами на языке ассемблера NASM (рис.
\ref{fig-001}).

\begin{figure}

\centering{

\includegraphics[width=0.6\linewidth,height=\textheight,keepaspectratio]{image/sp1.png}

}

\caption{\label{fig-001}Создание каталога}

\end{figure}%

2.Перехожу в созданный каталог (рис. \ref{fig-002}).

\begin{figure}

\centering{

\includegraphics[width=0.6\linewidth,height=\textheight,keepaspectratio]{image/sp2.png}

}

\caption{\label{fig-002}Переход в каталог}

\end{figure}%

3.Создаю текстовый файл с именем hello.asm (рис. \ref{fig-003}).

\begin{figure}

\centering{

\includegraphics[width=0.6\linewidth,height=\textheight,keepaspectratio]{image/sp3.png}

}

\caption{\label{fig-003}Создание текстового файла}

\end{figure}%

4.Открою этот файл с помощью текстового редактора Kate (рис.
\ref{fig-004}).

\begin{figure}

\centering{

\includegraphics[width=0.6\linewidth,height=\textheight,keepaspectratio]{image/sp4.png}

}

\caption{\label{fig-004}Открытие файла}

\end{figure}%

5.Ввожу в него следующий текст (рис. \ref{fig-005}).

\begin{figure}

\centering{

\includegraphics[width=0.6\linewidth,height=\textheight,keepaspectratio]{image/sp5.png}

}

\caption{\label{fig-005}Ввод текста}

\end{figure}%

6.Скомпилируем данный текст (рис. \ref{fig-006}).

\begin{figure}

\centering{

\includegraphics[width=0.6\linewidth,height=\textheight,keepaspectratio]{image/sp6.png}

}

\caption{\label{fig-006}Компиляция текста}

\end{figure}%

7.Проверю, что объектный файл был создан (рис. \ref{fig-007}).

\begin{figure}

\centering{

\includegraphics[width=0.6\linewidth,height=\textheight,keepaspectratio]{image/sp7.png}

}

\caption{\label{fig-007}Проверка наличия объектного файла}

\end{figure}%

8.Скомпилирую исходный файл hello.asm в obj.o и создадим файл листинга
list.lst (рис. \ref{fig-008}).

\begin{figure}

\centering{

\includegraphics[width=0.6\linewidth,height=\textheight,keepaspectratio]{image/sp8.png}

}

\caption{\label{fig-008}Создание файлов}

\end{figure}%

9.Проверю, что файлы были созданы (рис. \ref{fig-009}).

\begin{figure}

\centering{

\includegraphics[width=0.6\linewidth,height=\textheight,keepaspectratio]{image/sp9.png}

}

\caption{\label{fig-009}Проверка наличия файлов}

\end{figure}%

10.Передаю объектный файл на обработку компоновщику (рис.
\ref{fig-0010}).

\begin{figure}

\centering{

\includegraphics[width=0.6\linewidth,height=\textheight,keepaspectratio]{image/sp10.png}

}

\caption{\label{fig-0010}Передача файла компоновщику}

\end{figure}%

11.Проверю, что исполняемый файл hello был создан (рис. \ref{fig-0011}).

\begin{figure}

\centering{

\includegraphics[width=0.6\linewidth,height=\textheight,keepaspectratio]{image/sp11.png}

}

\caption{\label{fig-0011}Проверка наличия файла hello}

\end{figure}%

12.Задам имя создаваемого исполняемого файла (рис. \ref{fig-0012}).

\begin{figure}

\centering{

\includegraphics[width=0.6\linewidth,height=\textheight,keepaspectratio]{image/sp12.png}

}

\caption{\label{fig-0012}Задача имя исполняемого файла}

\end{figure}%

13.Запущу на выполнение созданный исполняемый файл, находящийся в
текущем каталоге (рис. \ref{fig-0013}).

\begin{figure}

\centering{

\includegraphics[width=0.6\linewidth,height=\textheight,keepaspectratio]{image/sp13.png}

}

\caption{\label{fig-0013}Запуск на выполнение созданный исполняемый
файл}

\end{figure}%

14.Создам копию файла hello.asm с именем lab04.asm (рис.
\ref{fig-0014}).

\begin{figure}

\centering{

\includegraphics[width=0.6\linewidth,height=\textheight,keepaspectratio]{image/sp14.png}

}

\caption{\label{fig-0014}Создание копии файла}

\end{figure}%

15.Проверяю, что копия создалась с именем lab04.asm (рис.
\ref{fig-0015}).

\begin{figure}

\centering{

\includegraphics[width=0.6\linewidth,height=\textheight,keepaspectratio]{image/sp15.png}

}

\caption{\label{fig-0015}Проверка создании копии файла}

\end{figure}%

16.Оттранслирую полученный текст программы lab04.asm в объектный файл.
Выполню компоновку объектного файла и запущу получившийся исполняемый
файл (рис. \ref{fig-0016}).

\begin{figure}

\centering{

\includegraphics[width=0.6\linewidth,height=\textheight,keepaspectratio]{image/sp16.png}

}

\caption{\label{fig-0016}Оттранслирование, компоновка, запуск}

\end{figure}%

17.Скопирую файлы hello.asm и lab04.asm в локальный репозиторий и
загружу файлы на Github.

\chapter{Выводы}\label{ux432ux44bux432ux43eux434ux44b}

В ходе выполнения работы, я освоила процедуры компиляции и сборки
программ, написанных на ассемблере NASM.

\#Список литературы

\begin{enumerate}
\def\labelenumi{\arabic{enumi}.}
\tightlist
\item
  GDB: The GNU Project Debugger. --- URL:
  https://www.gnu.org/software/gdb/.
\item
  GNU Bash Manual. --- 2016. --- URL:
  https://www.gnu.org/software/bash/manual/.
\item
  Midnight Commander Development Center. --- 2021. --- URL:
  https://midnight-commander. org/.
\item
  NASM Assembly Language Tutorials. --- 2021. --- URL:
  https://asmtutor.com/.
\item
  Newham C. Learning the bash Shell: Unix Shell Programming. ---
  O'Reilly Media, 2005. --- 354 с. --- (In a Nutshell). --- ISBN
  0596009658. --- URL:
  http://www.amazon.com/Learningbash-Shell-Programming-Nutshell/dp/0596009658.
\item
  Robbins A. Bash Pocket Reference. --- O'Reilly Media, 2016. --- 156 с.
  --- ISBN 978-1491941591.
\item
  The NASM documentation. --- 2021. --- URL:
  https://www.nasm.us/docs.php.
\item
  Zarrelli G. Mastering Bash. --- Packt Publishing, 2017. --- 502 с. ---
  ISBN 9781784396879.
\item
  Колдаев В. Д., Лупин С. А. Архитектура ЭВМ. --- М. : Форум, 2018.
\item
  Куляс О. Л., Никитин К. А. Курс программирования на ASSEMBLER. --- М.
  : Солон-Пресс,
\item
\item
  Новожилов О. П. Архитектура ЭВМ и систем. --- М. : Юрайт, 2016.
\item
  Расширенный ассемблер: NASM. --- 2021. --- URL:
  https://www.opennet.ru/docs/RUS/nasm/.
\item
  Робачевский А., Немнюгин С., Стесик О. Операционная система UNIX. ---
  2-е изд. --- БХВПетербург, 2010. --- 656 с. --- ISBN
  978-5-94157-538-1.
\item
  Столяров А. Программирование на языке ассемблера NASM для ОС Unix. ---
  2-е изд. --- М. : МАКС Пресс, 2011. --- URL:
  http://www.stolyarov.info/books/asm\_unix.
\item
  Таненбаум Э. Архитектура компьютера. --- 6-е изд. --- СПб. : Питер,
  2013. --- 874 с. --- (Классика Computer Science).
\item
  Таненбаум Э., Бос Х. Современные операционные системы. --- 4-е изд.
  --- СПб. : Питер,
\item
  --- 1120 с. --- (Классика Computer Science).
\end{enumerate}


\printbibliography



\end{document}
